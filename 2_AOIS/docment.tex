\documentclass[a4paper,11pt]{article}
\usepackage{cmap}
\usepackage{indentfirst}
\usepackage[warn]{mathtext}
\usepackage[T2A]{fontenc}
\usepackage[utf8]{inputenc}
\usepackage[argument]{graphicx}
\usepackage{tikz}  
\usetikzlibrary{graphs}
\usepackage{graphicx}
\usepackage{pgfplots}
\usepackage{tikz}
\usepackage{amscd}
\usepackage{longtable}
\usepackage[russian]{babel}
\usepackage{indentfirst}
\usepackage[left=1cm,right=1cm,top=1cm,bottom=1.5cm,bindingoffset=0cm]{geometry}
\author{Министерство образования Республики Беларусь \\ \\

Учреждение образования \\
“Белорусский государственный университет \\
информатики и радиоэлектроники” \\ \\ 

Факультет информационных технологий и управления \\ 
Кафедра интеллектуальных информационных технологий \\
 \\ \\ \\ \\ \\ \\ \\ \\ \\ \\ \\ \\ \\ \\ \\ \\ \\ \Large{\textbf{ЛАБОРАТОРНАЯ РАБОТА №1}} \\ \Large{по дисциплине « Алгоритмические основы \\ интеллектуальных систем» \\ \\ \\ \\  \\ \\ \\ \\ \\ \\ \\ \\ \\ \\ \\ \\ \\
 }
 \\ \\ \\ \\ \\ \\ \\ \\ \\ \\ \\ \\ \\ \\ \\}
\date{\begin{flushleft}
Выполнил (Самчук Е.А.) \\ студент группы \\ 021701 \\ \\ \\ \\ \\
Проверил (Жук А.А.)
\end{flushleft}Минск 2021}
\begin{document}
\maketitle
\begin{flushleft}
\Large{\textbf{Задание:} (Вариант 19,X_{1} = 13,X_{2} = 25) \\ 
    Составить и проверить программу, обеспечивающую выполнение
следующих задач:
    1. Перевод из десятичной системы счисления в двоичную и выполнение сложения/вычитания чисел Х1 и Х2 в прямом, дополнительном и обратном кодах всех вариантов слагаемых (+/+; +/-; -/+; -/-). \\
    2. Выполнить умножение модулей двух чисел Х1 и Х2 (значения чисел взять из соответствующих вариантов задания №1), определить знаки произведения для всех вариантов знаков сомножителей. \\
    3. Выполнить деление модуля числа Х1 на модуль числа Х2 (значения чисел взять из соответствующих вариантов задания №1). Результат округлить до 5 разрядов. Определить знаки частного для всех вариантов знаков делимого (Х1) и делителя (Х2).  \\ 
    4. Выполнить сложение двух чисел Х1 и Х2, представленных в форме с плавающей точкой. \\
Значения мантисс М1 и М2 взять из соответствующих вариантов значения чисел X_{1} и X_{2} задания №1.
Значения порядков (Р_{1} и Р_{2}) взять Р_{1}=0,100 , Р_{2}=0,101  для всех вариантов значений мантисс.}
\begin{center}
    \Large{\textbf{Описание программы}}
\end{center}
\Large{1.Операция бинарной суммы и разности в прямом коде:}В прямом коде бинарная сумма двух чисел производится по стандартному алгоритму операции сложения двух чисел в прямом коде:программа последовательно складывает соответствующие цифры первого и второго бинарных чисел (0 + 0 = 0,0 + 1 = 1 + 0 = 1),и результат заносится в соответствующую ячейку для цифры в результирующем бинарном числе: \\
\includegraphics[width = 14.0 cm]{Screenshot from 2021-09-23 19-52-31.png} \\
Если эти две цифры - единицы ,то в соответствующую ячейку заносится ноль и специальная переменная отражающая наличие дополнительной единицы (\_\_trans\_\_) становится равна true(правда).И на следующем этапе сложения цифр будет учитываться наличие дополнительной единицы из прошлой: \\
\includegraphics[width = 18.0 cm]{Screenshot from 2021-09-23 19-55-36.png} \\
Данный метод выполняется только в случае, если у нас два числа положительных. \\
В случае если имеются отрицательные числа,то существует 2 ситуации: \\
а)Если первое число и второе число отрицательные:в данном случае будет вызываться просто метод бинарной суммы описанный выше,число знака при этом сохранится: \\
\includegraphics[width = 18.0 cm]{Screenshot from 2021-09-23 20-03-33.png} \\
b)Если одно из чисел положительное:в случае если отрицательное - больше по модулю,то программа выполняет сложение на основе свойства:(a-b = - (b - a)): \\
\includegraphics[width = 19 cm]{Screenshot from 2021-09-23 20-12-02.png} \\
После этого производится стандартная операция вычитания бинарных чисел.Если вверху находится 0, а снизу 1,то алгоритм ищет единицу в старших разрядах( если единица находится в соседнем более высоком разряде - то она становится нулём,а в результат записывается 1,если единица находится дальше чем на 1 разряд - то она не становится нулём,а только в результат записывается 1):\\ 
\includegraphics[width = 19 cm]{Screenshot from 2021-09-23 20-19-57.png} \\ 
Так как алгоритм не трогает знаковые разряды,то в конце алгоритм проверяет каким должен быть результат и ставит в соответствующую ячейку цифру: \\
\includegraphics[]{Screenshot from 2021-09-23 20-24-38.png} \\
2.Операция бинарной суммы и разности для обратного кода: \\ a)В случае, если два числа положительных, то операция суммы будет такой же как и для прямого кода. \\
b)В случае, если модуль отрицательного числа больше положительного,то будем производиться такая же операция суммы, как и у прямого кода,только после вычисления суммы,результат будет переводиться из обратного в прямой код. \\
\includegraphics[width = 19 cm]{Screenshot from 2021-09-23 20-45-27.png} \\
c)В случае,если разность будет положительной,то будет производиться аналогичная сумма,после выполнения операции будет прибавлена бинарная единица: \\
\includegraphics[width = 10 cm]{Screenshot from 2021-09-23 20-47-07.png}
\\
d)В случае если два числа - отрицательные,то первоначально производится стандартная сумма,потом прибавляется единица к результату и результат переводится в прямой код: \\
\includegraphics[]{Screenshot from 2021-09-23 20-48-11.png} \\
3.Бинарная сумма для дополнительного кода: \\
а) В случае если два числа положительных - то производится обычная бинарная сумма.
b) В случае если разность положительная - то также производится обычная бинарная сумма из первого пункта.
с) В случае если разность отрицательна, но один из членов положительный, то производится бинарная сумма, прибавляется единица к результату,а также после этого результат переводится в прямой: \\
\includegraphics[width = 20 cm]{Screenshot from 2021-09-23 21-00-46.png} \\
4.Умножение бинарных чисел: \\
Умножение производится по стандартным правилам умножения бинарных чисел.Знак результата определяется из знаков исходных чисел.Берётся цифра второго множителя,если она равна единице,то к результату(первоначально равному нулю) прибавляется первый сомножитель с условием отступа на то количество разрядов,равное разряду цифры второго числа.После данных вычислений,так как знаковый разряд не трогается программой, будет на основе исходных данных выбираться знак результата: \\
\includegraphics[width = 20 cm]{Screenshot from 2021-09-23 21-16-45.png} \\
5.Бинарное деление: \\
Программа выполняет деление на основе правила деления "уголком".То есть создаётся цикл на 5 итераций(сколько надо найти разрядов после запятой по заданию),и в начале цикла автоматически происходит проверка на количество цифр в числе после запятой(т.к. далее в программе может добавиться больше, чем 1 цифра, за 1 итерацию).Если их 6,то программа выходит из цикла.Далее в цикле происходит добавление цифр в начало делимого,если оно меньше делителя, и также в результат добавляются нули,пока делимое не станет больше делителя.После того,как делимое стало больше делителя,происходит добавление 1 в результат и разность между делимым и делителем.Далее цикл повторяет действия:\\
\includegraphics[width = 20 cm]{Screenshot from 2021-09-24 19-04-45.png} \\
6.Сложение чисел с плавающей точкой: \\
7.Дополнительные функции для преобразования чисел: \\
а) Функция для перевода из десятичной в двоичную систему счисления в прямой код: Для перевода используется нахождение остатка от деления на 2:находится остаток 2,записывается в конец числа,само число нацело делится на 2 и действия повторяются до того как число не стало равно 0.После этого определяется знаковый разряд: \\
\includegraphics[]{Screenshot from 2021-09-24 21-00-58.png} \\
b) Функция для перевода в обратный код десятичного числа: первоначально число переводится в прямой код,если оно отрицательное то все цифры кроме знакового разряда меняется на противоположные: \\
\includegraphics[width = 18 cm]{Screenshot from 2021-09-24 21-03-43.png} \\
c) Функция перевода в дополнительный код: если число неотрицательное, то производится обычный перевод в прямой код,если отрицательное переводится в обратный код и прибавляется единица: \\
\includegraphics[width = 18 cm]{Screenshot from 2021-09-24 21-07-25.png} \\
d) Функции перевода из одного кода в другой: чтобы перевести из обратного в прямой код происходит инверсия разрядов числа при сохранении знакового разряда: \\
\includegraphics[width = 15 cm]{Screenshot from 2021-09-24 21-11-05.png} \\
Чтобы перевести из бинарного в десятичную в систему счисления значение разряда умножается на 2 степени номера разряда и находится сумма всех таких произведений: \\
\includegraphics[width = 15 cm]{Screenshot from 2021-09-24 21-17-21.png} \\
8.Функция вывода бинарного числа: выводит бинарное число: \\
\includegraphics[]{Screenshot from 2021-09-24 21-59-09.png} \\
\begin{center}
    \textbf{Выполнение программы}
\end{center}
В функции main находится оператор switch - case.То есть программа будет выглядеть ввиде консольного меню: \\
\includegraphics[width = 15 cm]{Screenshot from 2021-09-24 23-44-07.png} \\
Как видно из скрина,в программе 21 пункт меню,чтобы выйти из программы требуется ввести число ноль,программа после завершения выполнения пункта меню опять выводит данное консольное меню.В каждом пункте меню кроме последнего будет выводиться число в соответствующем коде. \\
Скрины вывода всех пунктов в меню: \\
\includegraphics[width = 16 cm,length = 7 cm]{Screenshot from 2021-09-24 23-55-43.png} \\
\includegraphics[width = 16 cm,length = 7 cm]{Screenshot from 2021-09-24 23-56-03.png} \\
\includegraphics[width = 16 cm,length = 7 cm]{Screenshot from 2021-09-24 23-56-17.png} \\
\includegraphics[width = 16 cm,length = 7 cm]{Screenshot from 2021-09-24 23-56-39.png} \\
\includegraphics[width = 16 cm,length = 7 cm]{Screenshot from 2021-09-24 23-56-56.png} \\
\includegraphics[width = 16 cm,length = 7 cm]{Screenshot from 2021-09-24 23-57-08.png} \\
\includegraphics[width = 16 cm,length = 7 cm]{Screenshot from 2021-09-24 23-57-23.png} \\
\end{flushleft}
\end{document}
